% https://github.com/martinhelso/OsloMet

\documentclass[UKenglish, aspectratio = 169]{beamer}
\usetheme{OsloMet}
\usepackage{style}

\author[Tomás \& Dylan]
{Tomás O'Malley \texorpdfstring{\\}{} Dylan Creavan}
\title{Galway Mayo Institute of Technology }
\subtitle{Turing Prize Winner Presentation on Frances Allen \texttt{}}

\begin{document}


\begin{frame}{Introduction }
		
	\begin{itemize}
		\item Welcome to our presentation on Frances Allen as part of the module Research methods.
		      We are currently fourth-year Software Development Students @ Galway-Mayo Institute of Technology attending the Dublin road Campus.  
		      		          
		\item We decided to dedicate our presentation to  Frances Allen the first woman to win the Turing Award for her contributions to modern computing who recently passed away. 
		\item This presentation was created using  Beamer Latex. Beamer is a LaTeX document class for creating presentation slides.
		      
		      		      
	\end{itemize}
		     
		     
		    
	\begin{columns}[T]
		\begin{column}{.5\textwidth}
			\begin{block}{Author}
				\begin{itemize}
					\item   Tomás O'Malley (G00361128)@gmit.ie
					\item Dylan Creavan (G00354442)@gmit.ie 
					  
				\end{itemize}    \end{block}
				\end{column}
				\begin{column}{.5\textwidth}
					\begin{block}{Topic}

					      		      
					\item	2006 Turing Prize Winner Frances Allen	 				
						
						
						
					\end{block}
				\end{column}
				\end{columns}
\end{frame}

% Use
%
%     \begin{frame}[allowframebreaks]
%
% if the TOC does not fit one frame.
\begin{frame}[allowframebreaks]
	\tableofcontents
\end{frame}



%% Disable the logo in the lower right corner:
\hidelogo


%% Enable the logo in the lower right corner:
\showlogo



\section{Who is Alan Turing and what is the Turing Award?}
\SectionPage

\begin{frame}[allowframebreaks]{Alan Turing}
\begin{columns}[T]
		\begin{column}{.5\textwidth}
			\begin{block}{Background }
				\begin{itemize}
					\itemAlan Turing was an English mathematician, who was highly influential in the development of theoretical computer science and was the creator of the concept of the Turing machine, which can be considered the template of the modern-day Personal Computer

					      		      
					\item He is best known though for his work during World War II.
					     
					     
					     
				\end{itemize}    \end{block}
				\end{column}
				\begin{column}{.5\textwidth}
					\begin{block}{Alan Turing}
						% Your image included here
						
						
						\includegraphics[height=0.5\textheight]{OM-images/alan.jpg}
						
							, Alan Turing age 16  
						
						
						
					\end{block}
				\end{column}
				\end{columns}
				
				
				
					\begin{itemize}
					\item
					     He worked for the Government Code and Cypher School at Bletchley Park, a codebreaking center that intercepted and decrypted German messages.
					      		      
					      		      
					\item
                             He is best known though for his work during World War II. He worked for the Government Code and Cypher School at Bletchley Park, a codebreaking center that intercepted and decrypted German messages.					     
					     
					  
					  \item 
					  
  Here, he created a codebreaking machine called "Bombe", a vast improvement upon its namesake, the Polish code-breaking machine, "bomba kryptologiczna". It is estimated that his efforts and accomplishments in code breaking and computer engineering shorted the war by two years and saved over 14 million lives.					    
					    
					    
					    
					    \item
 He died in 1954, at the age of 41 of what is officially a suicide from cyanide poisoning but has also been questioned as to the accidental inhalation of cyanide fumes from an apparatus used to electroplate gold onto spoons.
 \end{itemize}  
				
\end{frame}
\begin{frame}[allowframebreaks]{Turing Award}
		
	\begin{columns}[T]
		\begin{column}{.5\textwidth}
			\begin{block}{Background }
				\begin{itemize}
					
					      
					      		      
					\item The Turing Award is an annual prize given by the Association for Computing Machinery for contributions "of lasting and major technical importance to the computer field".
					\item The Award is, of course, named after Alan Turing and is generally recognized as the highest distinction in computer science or the "Nobel Prize of Computing".
				\end{itemize}    \end{block}
				\end{column}
				\begin{column}{.5\textwidth}
					\begin{block}{Turing Award photo}
						% Your image included here
						
						
			\includegraphics[height=0.5\textheight]{OM-images/turingAward.jpg}
						
							\item Turing Award photo 
						
					\end{block}
				\end{column}
				\end{columns}
				
				
				
								\begin{itemize}
                                
									\item The first recipient was Alan Perlis, in 1966.\newline
						
										\item  Intel also began funding the award in 2002 and is also now sponsored by the multinational company Google as of the year 2007. \newline
											\item Alongside the prestigious award recipients receive 1 million dollars for their breakthrough in computing.
										

                    \end{itemize}


					\end{frame}
\section{Who is Frances   Allen ?}
\SectionPage
\begin{frame}[allowframebreaks]{Frances Allen Background}
		
	\begin{columns}[T]
		\begin{column}{.5\textwidth}
			\begin{block}{Background }
				\begin{itemize}
					\item
   Birth: Frances E. Allen, (born August 4, 1932, Peru, New York, U.S.—died August 4, 2020), 					      		      
					      		      
					\item
					      Education  :  Allen received a bachelor’s degree (1954) in mathematics from Albany State Teachers College (now the State University of New York, Albany) and a master’s degree (1957) in mathematics from the University of Michigan.
				\end{itemize}    \end{block}
				\end{column}
				\begin{column}{.5\textwidth}
					\begin{block}{Frances Allen}
						% Your image included here
						
						
						\includegraphics[height=0.5\textheight]{OM-images/franchesPortrait.jpg}
						
							\item Frances Allen @ the 2006 Turing prize  
						
					\end{block}
				\end{column}
				\end{columns}
					
					
				
				
				\begin{itemize}
					\item
					      Career:  Allen During the 1960s Allen worked on compilers for IBM supercomputers, such as the IBM 7030 (known as Stretch) and the IBM 7950 (known as Harvest), which were ordered by the U.S. National Security Agency for delivery to Los Alamos National Laboratory. Much of her subsequent work concerned efficient computer programming for multiprocessing systems,
					      
					\item  Frances Allen is widely known  	For her contributions to program optimization and compiling for parallel computers.

				\end{itemize}
				
				
				
				
				\end{frame}
				
				\section{Career and Research}
				\SectionPage

	\begin{frame}[allowframebreaks]{Career and Research}
					
					

					\begin{itemize}
						
								
				\item After graduating from the University of Michigan in 1957, she began teaching the company's researchers Fortran, the first high-level programming language. She planned to work at the company until she paid off her student loans, but she ended up staying at IBM for her entire 45-year career, retiring in 2002.

				\item Frances is widely known for her contributions to program optimization and compiling for parallel computers.

						
					\end{itemize}
				\end{frame}





	\section{Contribution to Computing}
				\SectionPage
	\begin{frame}[allowframebreaks]{Contribution to Computing}
					
					

					\begin{itemize}
						
								
				\item In 1989, she became the first woman to be named an IBM Fellow and, in 2006, received the ACM Turing Award.

                \item Allen began her career at a small rural high school in Peru, New York, teaching practical math to farm kids, then took a job at IBM to earn the money she needed to pay off her college loans. 
                
                \item  Frances Allen spent 45 years at IBM
								
				\item An IBM Fellow is an appointed position at IBM made by IBM's CEO. Typically only four to nine (eleven in 2014) IBM Fellows are appointed each year, in May or June. The fellow is the highest honor a scientist, engineer, or programmer at IBM can achieve. 

						
					\end{itemize}
				\end{frame}
				\begin{frame}{Contribution to Computing II - Why did she win a Turing Award?}
					
					

					\begin{itemize}
						
				\item She made fundamental contributions to the theory and practice of program optimization
				
				\item Her contributions also greatly extended earlier work in  automatic program parallelization
				
				\item  These techniques have made it possible to achieve high performance from computers whilst programming.
				\item Allen devoted her career to compiler optimization which was sparked by her work at IBM teaching the brand-new FORTRAN programming language to IBM employees.
				\item Her first major project in this way for the Stretch-Harvest computer in the early 1960s, designed to handle top-secret code-breaking and intelligence for the National Security Agency.
						
					\end{itemize}
				\end{frame}
				\section{Questions}
		
				\SectionPage
					
			
\begin{frame}[allowframebreaks]{References}
    \begin{thebibliography}{}

        % Article is the default.
        \setbeamertemplate{bibliography item}[triangle]


        \bibitem{Biography}
        Bio
        \newblock \emph{Frances Allen Bio}, 2020.
        \newblock \url{https://www.britannica.com/biography/Frances-E-Allen}
        
        
         \bibitem{Biography}
        Birthplace
        \newblock \emph{Frances Allen Birthplace}, 2020.
        \newblock \url{https://en.wikipedia.org/wiki/Frances_Allen}
        
        
         \bibitem{Biography}
        Reasons for winning the Turin Award 1
        \newblock \emph{Frances Allen Turing Award }, 2020.
        \newblock \url{https://www.acm.org/media-center/2007/fran-allen-wins-2006-a-m-turing-award}
        
        
         \bibitem{Biography}
        Reasons for winning the Turin Award 2
        \newblock \emph{Frances Allen Turing Award }, 2020.
        \newblock \url{https://www.britannica.com/biography/Frances-E-Allen}
        
        
        
         \bibitem{Biography}
        Frances Allen's obituary in the Washington Post
        \newblock \emph{Frances Allen Turing Award }, 2020.
        \newblock \url{https://www.washingtonpost.com/local/obituaries/frances-allen-first-woman-to-win-turing-award-for-contributions-to-computing-dies-at-88/2020/08/06/7ea7d7a2-d7f0-11ea-930e-d88518c57dcc_story.html}
        
        


        \bibitem{Biography}
        IBM - Remembering Frances Allen
        \newblock \emph{Frances Allen Turing Award }, 2020.
        \newblock \url{https://www.ibm.com/blogs/research/2020/08/remembering-frances-allen/}

        
        \bibitem{Beamer}
        Beamer Latex Template
        \newblock \emph{Beamer  }, 2020.
        \newblock \url{https://en.wikipedia.org/wiki/Beamer_(LaTeX)}
        
        
%	https://www.ibm.com/blogs/research/2020/08/remembering-frances-allen/#:~:text=IBM%20Fellow%20Fran%20Allen%20spent,day%20of%20her%2088th%20birthday.


   

    \end{thebibliography}
\end{frame}
				
				
\end{document}